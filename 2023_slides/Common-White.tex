% \usepackage{dbt}
\setbeamercovered{dynamic}
\usepackage{etex}
\usepackage{multirow}
\usepackage{amsmath}
\usepackage{graphicx,url,psfrag}
\usepackage{tikz}
\usetikzlibrary{
  arrows,
  arrows.meta,
  calc,
  decorations.fractals,
  decorations.pathreplacing,
  graphs,
  mindmap,
  positioning,
  shapes.geometric,
  through,shapes,patterns,
  trees,
  positioning
  }
\usepackage{smartdiagram}
\usesmartdiagramlibrary{additions}
\usepackage[center]{subfigure}
\usepackage{enumerate}
\usepackage[makeroom]{cancel}
\usepackage{mathtools}
\usepackage{graphbox}
\usepackage{amssymb}
\usepackage{amsfonts}
\usepackage{mathrsfs}
\usepackage{comment}
\usepackage{xfrac}
\usepackage{pgfplots}
\excludecomment{codes}
% \usepackage{movie15}
% \usepackage[showframe]{geometry}
% \usepackage{enumitem}
\usepackage{pdfpages}

%
% for warning sign
%
\usepackage{pgfplots}
\usepackage{stackengine}
\usepackage{scalerel}
\usepackage{xcolor}
\newcommand\dangersign[1][2ex]{%
  \renewcommand\stacktype{L}%
  \scaleto{\stackon[1.3pt]{\color{red}$\triangle$}{\tiny !}}{#1}%
}
% %  The following is to show codes:
\usepackage{listings}
% \usepackage{color}

\usepackage{poker}
\setkeys{poker}{inline=symbol}
\usepackage{epsdice}

\definecolor{dkgreen}{rgb}{0,0.6,0}
\definecolor{gray}{rgb}{0.5,0.5,0.5}
\definecolor{mauve}{rgb}{0.58,0,0.82}

\lstset{frame=tb,
  language=Java,
  aboveskip=3mm,
  belowskip=3mm,
  showstringspaces=false,
  columns=flexible,
  basicstyle={\small\ttfamily},
  numbers=none,
  numberstyle=\tiny\color{gray},
  keywordstyle=\color{blue},
  commentstyle=\color{dkgreen},
  stringstyle=\color{mauve},
  breaklines=true,
  breakatwhitespace=true,
  tabsize=3
}
\lstset{language=Python}

\lstset{ %
  language=Python,                     % the language of the code
  basicstyle=\footnotesize,       % the size of the fonts that are used for the code
  numbers=left,                   % where to put the line-numbers
  numberstyle=\tiny\color{gray},  % the style that is used for the line-numbers
  stepnumber=1,                   % the step between two line-numbers. If it's 1, each line
                                  % will be numbered
  numbersep=5pt,                  % how far the line-numbers are from the code
  backgroundcolor=\color{white},  % choose the background color. You must add \usepackage{color}
  showspaces=false,               % show spaces adding particular underscores
  showstringspaces=false,         % underline spaces within strings
  showtabs=false,                 % show tabs within strings adding particular underscores
  frame=single,                   % adds a frame around the code
  rulecolor=\color{black},        % if not set, the frame-color may be changed on line-breaks within not-black text (e.g. commens (green here))
  tabsize=2,                      % sets default tabsize to 2 spaces
  captionpos=b,                   % sets the caption-position to bottom
  breaklines=true,                % sets automatic line breaking
  breakatwhitespace=false,        % sets if automatic breaks should only happen at whitespace
  title=\lstname,                 % show the filename of files included with \lstinputlisting;
                                  % also try caption instead of title
  keywordstyle=\color{blue},      % keyword style
  commentstyle=\color{dkgreen},   % comment style
  stringstyle=\color{mauve},      % string literal style
  escapeinside={\%*}{*)},         % if you want to add a comment within your code
  morekeywords={*,...}            % if you want to add more keywords to the set
}
% \usepackage[usenames,dvipsnames]{color}
% \lstset{
%   language=R,                     % the language of the code
%   basicstyle=\tiny\ttfamily, % the size of the fonts that are used for the code
%   numbers=left,                   % where to put the line-numbers
%   numberstyle=\tiny\color{Blue},  % the style that is used for the line-numbers
%   stepnumber=1,                   % the step between two line-numbers. If it is 1, each line
%                                   % will be numbered
%   numbersep=5pt,                  % how far the line-numbers are from the code
%   backgroundcolor=\color{white},  % choose the background color. You must add \usepackage{color}
%   showspaces=false,               % show spaces adding particular underscores
%   showstringspaces=false,         % underline spaces within strings
%   showtabs=false,                 % show tabs within strings adding particular underscores
%   frame=single,                   % adds a frame around the code
%   rulecolor=\color{black},        % if not set, the frame-color may be changed on line-breaks within not-black text (e.g. commens (green here))
%   tabsize=2,                      % sets default tabsize to 2 spaces
%   captionpos=b,                   % sets the caption-position to bottom
%   breaklines=true,                % sets automatic line breaking
%   breakatwhitespace=false,        % sets if automatic breaks should only happen at whitespace
%   keywordstyle=\color{RoyalBlue},      % keyword style
%   commentstyle=\color{YellowGreen},   % comment style
%   stringstyle=\color{ForestGreen}      % string literal style
% }

% \usepackage[dvipsnames]{xcolor}
% \newcommand{\Cross}{\mathbin{\tikz [x=1.4ex,y=1.4ex,line width=.2ex] \draw (0,0) -- (1,1) (0,1) -- (1,0);}}%
\newcommand{\Crossme}[1]{\!\!
\tikz [black,x=1.1em,y=1.1em,line width=.4ex]
\draw (-0.5,-0.5) -- (0,0) node {\footnotesize #1} -- (0.5,0.5) (0.5,-0.5) -- (-0.5,0.5);}%
\newcommand{\Checkme}[1]{\!\!
\tikz [x=1.1em,y=1.1em,line width=.4ex]
\draw [black] (0,0.7) -- (0.3,0) --(0.9,1.0) (0.5,0.5) node {\footnotesize #1};}
% \beamerdefaultoverlayspecification{<+-| alert@+>} %(this will show line by line)
\beamerdefaultoverlayspecification{<+->} %(this will show line by

% \usepackage{natbib}
% \input{../myMathSymbols.tex}
% \newcommand{\tlMr}[4]{\:{}^{\hspace{0.2em}#1}_{#2} \hspace{-0.1em}#3_{#4}}

% Smiley face\Smiley{} \Frowny{}
\usepackage{marvosym}
% -------------------------------------------------
%  Set directory for figs
% -------------------------------------------------
\usepackage{grffile}
\graphicspath{{Codes/}}
% -------------------------------------------------
%  Define colors
% -------------------------------------------------
\def\refcolor{cyan}
\def\excolor{brown}
% \usepackage{color}
% \usepackage[dvipsnames]{xcolor}


% % % Define danger sign
\newcommand*{\TakeFourierOrnament}[1]{{%
\fontencoding{U}\fontfamily{futs}\selectfont\char#1}}
\newcommand*{\danger}{\TakeFourierOrnament{66}}


% -------------------------------------------------
%  Define short-hand symbols.
% -------------------------------------------------
\newcommand{\B}{\textbf{B}}
\newcommand{\PP}{\mathbb{P}}
\newcommand{\E}{\mathbb{E}}
\newcommand{\D}{\mathbb{D}}
\newcommand{\W}{\dot{W}}
\newcommand{\ud}{\ensuremath{\mathrm{d}}}
\newcommand{\Ceil}[1]{\left\lceil #1 \right\rceil}
\newcommand{\Floor}[1]{\left\lfloor #1 \right\rfloor}
\newcommand{\sgn}{\text{sgn}}
\newcommand{\Lad}{\text{L}_{\text{ad}}^2}
\newcommand{\SI}[1]{\mathcal{I}\left[#1 \right]}
\newcommand{\SIB}[2]{\mathcal{I}_{#2}\left[#1 \right]}
\newcommand{\Indt}[1]{1_{\left\{#1 \right\}}}
\newcommand{\LadInPrd}[1]{\left\langle #1 \right\rangle_{\text{L}_\text{ad}^2}}
\newcommand{\LadNorm}[1]{\left|\left|  #1 \right|\right|_{\text{L}_\text{ad}^2}}
\newcommand{\Norm}[1]{\left|\left|  #1   \right|\right|}
\newcommand{\Ito}{It\^{o} }
\newcommand{\Itos}{It\^{o}'s }
\newcommand{\spt}[1]{\text{supp}\left(#1\right)}
\newcommand{\InPrd}[1]{\left\langle #1 \right\rangle}
\newcommand{\mr}{\textbf{r}}
\newcommand{\Ei}{\text{Ei}}
\newcommand{\arctanh}{\operatorname{arctanh}}
\newcommand{\ind}[1]{\mathbb{I}_{\left\{ {#1} \right\} }}
\newcommand{\Var}{\text{Var}}
\newcommand{\Cov}{\text{Cov}}
\newcommand{\Corr}{\text{Corr}}

\newcommand{\baseurl}[1]{\footnotesize\url{http://math.emory.edu/~lchen41/teaching/2020_Spring/#1}}

\def\bD{\mathbb{D}}
\def\bR{\mathbb{R}}
\def\bC{\mathbb{C}}
\def\e{\varepsilon}
\def\E{\mathbb{E}}
\def\P{\mathbb{P}}

\DeclareMathOperator{\esssup}{\ensuremath{ess\,sup}}

\newcommand{\steps}[1]{\vskip 0.3cm \textbf{#1}}
\newcommand{\calB}{\mathcal{B}}
\newcommand{\calC}{\mathcal{C}}
\newcommand{\calD}{\mathcal{D}}
\newcommand{\calE}{\mathcal{E}}
\newcommand{\calF}{\mathcal{F}}
\newcommand{\calG}{\mathcal{G}}
\newcommand{\calK}{\mathcal{K}}
\newcommand{\calH}{\mathcal{H}}
\newcommand{\calI}{\mathcal{I}}
\newcommand{\calL}{\mathcal{L}}
\newcommand{\calM}{\mathcal{M}}
\newcommand{\calN}{\mathcal{N}}
\newcommand{\calO}{\mathcal{O}}
\newcommand{\calT}{\mathcal{T}}
\newcommand{\calP}{\mathcal{P}}
\newcommand{\calR}{\mathcal{R}}
\newcommand{\calS}{\mathcal{S}}
\newcommand{\calV}{\mathcal{V}}
\newcommand{\bbC}{\mathbb{C}}
\newcommand{\bbN}{\mathbb{N}}
\newcommand{\bbP}{\mathbb{P}}
\newcommand{\bbZ}{\mathbb{Z}}
\newcommand{\myVec}[1]{\overrightarrow{#1}}
\newcommand{\sincos}{\begin{array}{c} \cos \\ \sin \end{array}\!\!}
\newcommand{\CvBc}[1]{\left\{\:#1\:\right\}}
\newcommand*{\one}{{{\rm 1\mkern-1.5mu}\!{\rm I}}}
\def\e{{\rm e}}
\def\cA{\mathcal{A}}
\def\cB{\mathcal{B}}
\def\cC{\mathcal{C}}
\def\cD{\mathcal{D}}
\def\cE{\mathcal{E}}
\def\cF{\mathcal{F}}
\def\cG{\mathcal{G}}
\def\cH{\mathcal{H}}
\def\cI{\mathcal{I}}
\def\cL{\mathcal{L}}
\def\cM{\mathcal{M}}
\def\cP{\mathcal{P}}
\def\cQ{\mathcal{Q}}
\def\cS{\mathcal{S}}
\def\cU{\mathcal{U}}

\newcommand{\OneFrame}[1]{
\begin{enumerate}\item[#1] \phantom{av} \\[20em]\vfill\phantom{av}\myEnd\end{enumerate}}

\newcommand{\bH}{\ensuremath{\mathrm{H}}}
\newcommand{\Ai}{\ensuremath{\mathrm{Ai}}}

\newcommand{\R}{\mathbb{R}}
\newcommand{\myEnd}{\hfill$\square$}
\newcommand{\ds}{\displaystyle}
\newcommand{\Shi}{\text{Shi}}
\newcommand{\Chi}{\text{Chi}}
\newcommand{\Erf}{\ensuremath{\mathrm{erf}}}
\newcommand{\Erfc}{\ensuremath{\mathrm{erfc}}}
\newcommand{\He}{\ensuremath{\mathrm{He}}}
\newcommand{\Res}{\ensuremath{\mathrm{Res}}}

\newcommand{\mySeparateLine}{\begin{center}
 \makebox[\linewidth]{\rule{0.6\paperwidth}{0.4pt}}
\end{center}}
\newcommand{\myfootnoteline}{\noindent\rule{0.3\textwidth}{0.4pt}\\ \bigskip}
\newcommand*{\TakeFourierOrnament}[1]{{%
\fontencoding{U}\fontfamily{futs}\selectfont\char#1}}
\newcommand*{\danger}{\TakeFourierOrnament{66}}

\theoremstyle{definition}
% \newtheorem{definition}[theorem]{Definition}
% \newtheorem{hypothesis}[theorem]{Hypothesis}
\newtheorem{assumption}[theorem]{Assumption}

\theoremstyle{plain}
% \newtheorem{theorem}{Theorem}
% \newtheorem{corollary}[theorem]{Corollary}
% \newtheorem{lemma}[theorem]{Lemma}
\newtheorem{proposition}[theorem]{Proposition}

\mode<presentation>
{
%      \usetheme{Warsaw}
%     \usetheme{JuanLesPins}
%  \usetheme{Hannover}
%  \usetheme{Montpellier}
   \useoutertheme{default}
  % or ...

  \setbeamercovered{transparent}
  % or whatever (possibly just delete it)
 \setbeamertemplate{frametitle}{
  \begin{centering}
    \color{blue}
    {\insertframetitle}
    \par
  \end{centering}
  }
}
\usefoottemplate{\hfill \insertframenumber{}}
% \inserttotalframenumber

\usepackage[english]{babel}
% or whatever

% \usepackage[latin1]{inputenc}
% or whatever

\usepackage{times}
\usepackage[T1]{fontenc}
% Or whatever. Note that the encoding and the font should match. If T1
% does not look nice, try deleting the line with the fontenc.

% \DeclareMathOperator{\Lip}{Lip}
\DeclareMathOperator{\lip}{l}
% \DeclareMathOperator{\Vip}{\overline{v}}
% \DeclareMathOperator{\vip}{\underline{v}}
% \DeclareMathOperator{\vv}{v}
% \DeclareMathOperator{\BC}{BC}
% \DeclareMathOperator{\CH}{CD}

\usepackage{pgfpages}
% \setbeameroption{show notes}
% \setbeamertemplate{note page}[plain]
% \setbeameroption{second mode text on second screen=right}
% \setbeameroption{show notes on second screen=right}
%

% Delete this, if you do not want the table of contents to pop up at
% the beginning of each subsection:
% \AtBeginSubsection[]
% {
%   \begin{frame}<beamer>{Outline}
%     \tableofcontents[currentsection,currentsubsection]
%   \end{frame}
% }

% If you wish to uncover everything in a step-wise fashion, uncomment
% the following command:
% \beamerdefaultoverlayspecification{<+->}

% % % % % % % % % % % % % % % % % % %
%  Define a block
% % % % % % % % % % % % % % % % % % %
\newenvironment<>{problock}[1]{%
  \begin{actionenv}#2%
      \def\insertblocktitle{#1}%
      \par%
      \mode<presentation>{%
        \setbeamercolor{block title}{fg=white,bg=olive!95!black}
       \setbeamercolor{block body}{fg=black,bg=olive!25!white}
       \setbeamercolor{itemize item}{fg=white!20!white}
       \setbeamertemplate{itemize item}[triangle]
     }%
      \usebeamertemplate{block begin}}
    {\par\usebeamertemplate{block end}\end{actionenv}}

\newenvironment<>{assblock}[1]{%
  \begin{actionenv}#2%
      \def\insertblocktitle{#1}%
      \par%
      \mode<presentation>{%
        \setbeamercolor{block title}{fg=white,bg=green!50!black}
       \setbeamercolor{block body}{fg=black,bg=green!10}
       \setbeamercolor{itemize item}{fg=green!80!black}
       \setbeamertemplate{itemize item}[triangle]
     }%
      \usebeamertemplate{block begin}}
    {\par\usebeamertemplate{block end}\end{actionenv}}


% \newtheorem{proofnoend}{Proof.}
% \AtBeginEnvironment{proofnoend}{%
%   \setbeamercolor{block title}{use=example text,fg=lgtblue,bg=background}
%   % \setbeamercolor{block body}{parent=normal text,use=block title example,fg=yellow}
% }

% Define some colors
\definecolor{white}{HTML}{FFFFFF}              % #FFFFFF
\definecolor{pink}{HTML}{FB73BE}               % #FB73BE
\definecolor{coral}{HTML}{FF8D71}              % #FF8D71
% \definecolor{yellow}{HTML}{9A7D0A}             % #9A7D0A
\definecolor{yellow}{HTML}{6E2C00}             % #6E2C00
% \definecolor{yellow}{HTML}{7D6608}             % #7D6608
% \definecolor{yellow}{HTML}{7E5109}             % #7E5109
% \definecolor{yellow}{HTML}{FFE066}             % #FFE066
\definecolor{teal}{HTML}{59F3CE}               % #59F3CE
\definecolor{lgtblue}{HTML}{2980B9} 	       % #1A5276
% \definecolor{lgtblue}{HTML}{65D0FA} 	       % #65D0FA
\definecolor{blue}{HTML}{4984F2}               % #4984F2
\definecolor{purple}{HTML}{A87DFF}             % #A87DFF
\definecolor{red}{HTML}{FF3d30}                % #FF3d30
% \definecolor{magenta}{HTML}{FF80FF}                % #FF3d30
% \definecolor{green}{HTML}{145A32}              % #145A32
\definecolor{green}{HTML}{1D8348}              % #145A32
% \definecolor{green}{HTML}{59F3CE}              % #59F3CE
\setbeamercolor{alerted text}{fg=red}
% \setbeamercolor{block title}{bg=background,fg=lgtblue}

\setbeamercolor{section in toc}{fg=black}
\setbeamercolor{subsection in toc}{fg=red}

% \newtheorem{myexample}{\it Example}[section]
\newcounter{myexample}[section]
\resetcounteronoverlays{myexample}
\newenvironment{myexample}[1][]{\refstepcounter{myexample}\par\medskip
\noindent \textbf{\textcolor{green}{Example~\mySecNum-\themyexample~#1}} \rmfamily}{\medskip}

\newcounter{mydefinition}[section]
\resetcounteronoverlays{mydefinition}
\newenvironment{mydefinition}[1][]{\refstepcounter{mydefinition}\par\medskip
\noindent \textbf{\textcolor{yellow}{Definition~\mySecNum-\themydefinition~#1}} \rmfamily}{\medskip}

% \NewCommandCopy{\oldref}{\ref}
% \let\oldref\ref
\newcommand{\myref}[1]{\mySecNum-\ref{#1}}

\newcounter{remark}[section]
\resetcounteronoverlays{remark}
\newenvironment{remark}[1][]{\refstepcounter{remark}\par\medskip
\noindent \textbf{\textcolor{blue}{Remark~\mySecNum-\theremark~#1}} \rmfamily}{\medskip}

\newcounter{mythm}[section]
\resetcounteronoverlays{mythm}
\newenvironment{mythm}[1][]{\refstepcounter{mythm}\par\medskip
\noindent \textbf{\textcolor{lgtblue}{Theorem~\mySecNum-\themythm~#1}} \rmfamily}{\medskip}

\newenvironment{mycor}[1][]{\refstepcounter{mythm}\par\medskip
\noindent \textbf{\textcolor{lgtblue}{Corollary~\mySecNum-\themythm~#1}} \rmfamily}{\medskip}

% \newtheorem{solution}{\textcolor{purple}{Solution}}
\newenvironment{mysol}[1][]{\par\medskip
\noindent \textbf{\textcolor{purple}{Solution#1.~}} \rmfamily}{\medskip}

\long\def\script#1{}

\usepackage{etoolbox}
\usepackage{ifthen}

% Put this file into the preamble and call it via
% \input{tikzdice.tex}
%
% Jesse Hamner
% https://github.com/jessehamner/tikzdice
% 2020


\usetikzlibrary{arrows}

\tikzset{
  treenode/.style = {align=center, text centered,
    font=\sffamily},
  arn_n/.style = {treenode, font=\sffamily\bfseries, draw=black,
    fill=black, text width=1.5em},
  arn_r/.style = {treenode, text width=1.5em},
  arn_x/.style = {treenode, minimum width=0.5em, minimum height=0.5em}
}

\newcommand{\majelaxis}{0.5}
\newcommand{\minelaxis}{0.125}

\newcommand{\headtext}{\color{red}\textbf{H}}

\newcommand{\dieone}{\maindie{1}{0}}
\newcommand{\dietwo}{\maindie{2}{0}}
\newcommand{\diethree}{\maindie{3}{0}}
\newcommand{\diefour}{\maindie{4}{0}}
\newcommand{\diefive}{\maindie{5}{0}}
\newcommand{\diesix}{\maindie{6}{0}}

\newcommand{\pipsize}{3.05pt}
\newcommand{\diecolor}{red}
\newcommand{\pipcolor}{red}
\newcommand{\fieldfill}{white}

\newcommand{\piplocation}[2]{\filldraw[fill=\pipcolor] (#1,#2) circle (\pipsize);}

\newcommand{\piplocationone}{\piplocation{0.25}{0.25}}
\newcommand{\piplocationtwo}{\piplocation{0.50}{0.25}}
\newcommand{\piplocationthree}{\piplocation{0.75}{0.25}}
\newcommand{\piplocationfour}{\piplocation{0.25}{0.50}}
\newcommand{\piplocationfive}{\piplocation{0.50}{0.50}}
\newcommand{\piplocationsix}{\piplocation{0.75}{0.50}}
\newcommand{\piplocationseven}{\piplocation{0.25}{0.75}}
\newcommand{\piplocationeight}{\piplocation{0.50}{0.75}}
\newcommand{\piplocationnine}{\piplocation{0.75}{0.75}}

\newcommand{\spreadsix}{\piplocation{0.20}{0.20}\piplocation{0.20}{0.80}
						\piplocation{0.80}{0.20}\piplocation{0.80}{0.80}
						\piplocation{0.50}{0.20}\piplocation{0.50}{0.80}}

\newcommand{\maindie}[2]{
\ifthenelse{#2 = 0}{\renewcommand{\diecolor}{black}\renewcommand{\pipcolor}{black}\renewcommand{\fieldfill}{white}}{}%
\ifthenelse{#2 = 1}{\renewcommand{\diecolor}{red}\renewcommand{\pipcolor}{red}\renewcommand{\fieldfill}{white}}{}%
\ifthenelse{#2 = 2}{\renewcommand{\diecolor}{blue}\renewcommand{\pipcolor}{blue}\renewcommand{\fieldfill}{white}}{}%
\ifthenelse{#2 = 3}{\renewcommand{\diecolor}{black}\renewcommand{\pipcolor}{white}\renewcommand{\fieldfill}{black}}{}%
\begin{tikzpicture}
  \draw[very thick, rounded corners, \diecolor, fill=\fieldfill] (0,0) rectangle (1,1);
  \ifthenelse{#1 = 1}{\piplocationfive}{}%
  \ifthenelse{#1 = 2}{\piplocationone\piplocationnine}{}%
  \ifthenelse{#1 = 3}{\piplocationone\piplocationfive\piplocationnine}{}%
  \ifthenelse{#1 = 4}{\piplocationone\piplocationnine\piplocationthree\piplocationseven}{}%
  \ifthenelse{#1 = 5}{\piplocationone\piplocationnine\piplocationthree\piplocationseven\piplocationfive}{}%
  \ifthenelse{#1 = 6}{\piplocation{0.20}{0.25}\piplocation{0.20}{0.75}
					  \piplocation{0.80}{0.25}\piplocation{0.80}{0.75}
					  \piplocation{0.20}{0.50}\piplocation{0.80}{0.50}}{}%

  \ifthenelse{#1 = 7}{\spreadsix\piplocationfive}{}%
  \ifthenelse{#1 = 8}{\spreadsix\piplocation{0.20}{0.50}\piplocation{0.80}{0.50}}{}%
  \ifthenelse{#1 = 9}{\spreadsix\piplocation{0.20}{0.50}\piplocation{0.80}{0.50}\piplocationfive}{}%
  \ifthenelse{#1 = 0}{}{}%
%    \ifthenelse{#1 = 0}{\draw[very thick, rounded corners, \diecolor, fill=\diecolor!20] (0,0) rectangle (1,1);}{}%

\end{tikzpicture}
}


\pgfmathdeclarefunction{addone}{1}{
  \pgfmathparse{#1 + \majelaxis}
}

\pgfmathdeclarefunction{addtwo}{1}{
  \pgfmathparse{#1 + \majelaxis + \majelaxis}
}

\pgfmathdeclarefunction{addspace}{1}{
  \pgfmathparse{#1 + 1.15}
}


\newcommand{\coin}[4]{
  \pgfmathsetmacro{\argthree}{addone(#1)}
  \pgfmathsetmacro{\argtwo}{addtwo(#1)}

  \shade[left color=white, right color=black]
    ({\argthree},0mm) ellipse (\majelaxis cm and \minelaxis cm);
  \shade[left color=white, right color=black]
    ({#1},0mm) rectangle (\argtwo cm, 1mm);

  \draw({\argthree},0.5cm) node[color=blue, above]{#2};

  \draw[fill=#4]({\argthree},1mm) ellipse (\majelaxis cm and \minelaxis cm);
  \draw ({#1},0mm) arc (180:360:\majelaxis cm and \minelaxis cm);
  \draw({#1},0mm) -- ({#1}, 1mm);
  \draw({\argtwo},0mm) -- ({\argtwo}, 1mm);

  \draw({\argthree},-0.4cm) node[color=black, below]{#3};
}

\newcommand{\heads}[2]{
\coin{#1}{#2}{\headtext}{blue!50}
}

\newcommand{\tails}[2]{
\coin{#1}{#2}{{\color{gray!70}T}}{white}
}




\newcommand{\wtf}{\theheadpos & \makecoinrow{} \\}

\newcommand{\whichheads}{1}
\newcounter{num}
\setcounter{num}{1}
\newcounter{headpos}
\setcounter{headpos}{1}
\newcounter{int}
\setcounter{int}{1}


% Row of coins, one head, slot #1.

\newcommand{\makecoinrow}{
  \setcounter{int}{1}

  \begin{tikzpicture}[]
    \newdimen\R
    \R=0cm

    \loop
      \ifthenelse{\value{int}=\value{headpos}}{\heads{\R}{}}{\tails{\R}{}}
      \pgfmathsetmacro{\R}{addspace(\R)}
      \stepcounter{int}
      \ifnum \value{int}<11
    \repeat

  \end{tikzpicture}

  \smallskip
}



% Definition of circles and square
\def\firstcircle{(0,0) circle (1.5cm)}
\def\secondcircle{(2,0) circle (1.5cm)}
\def\firstsquare{(-1.6,-1.6) rectangle (1.6,1.6)}


\def\missed{\LARGE\textbf{X}}
\def\target{\begin{tikzpicture}
  \draw[ultra thick, black, fill=blue!55] (0,0) circle(1cm);
  \draw[fill=red!70] (0,0) circle(0.73cm);
  \draw[fill=yellow] (0,0) circle(0.35cm) node{\small$\times$};
\end{tikzpicture}}

% Set colors
\colorlet{circle edge}{blue!50}
\colorlet{circle area}{blue!20}
\colorlet{rectangle edge}{blue!50}
\colorlet{rectangle area}{blue!20}

\tikzset{filled/.style={fill=circle area, draw=circle edge, thick},
    	 outline/.style={draw=circle edge, thick}
	    }
% This line makes makes theorems and boxed environments align incorrectly (vertically) in Beamer, I suggest that you remove it
% \setlength{\parskip}{5mm}

% This command is a nice shortcut in my opinion that makes the dice aligned vertically when they're used with text or in math mode.
% Possibly you want to use another (optional?) argument for the scalebox parameter to tune the size
\newcommand\vcdice[2]{\vcenter{\hbox{\scalebox{0.4}{\maindie{#1}{#2}}}}}



\def\missed{\LARGE\textbf{X}}
\def\target{\begin{tikzpicture}
  \draw[ultra thick, black, fill=blue!55] (0,0) circle(1cm);
  \draw[fill=red!70] (0,0) circle(0.73cm);
  \draw[fill=yellow] (0,0) circle(0.35cm) node{\small$\times$};
\end{tikzpicture}}

\tikzset{
  treenode/.style = {align=center, text centered, font=\sffamily},
  arn_n/.style = {treenode, font=\sffamily\bfseries, draw=black, fill=black},
  arn_r/.style = {treenode},
  arn_x/.style = {treenode}
}


% {{{ The following is for git information
\usepackage{gitinfo2}
% }}}

\usetikzlibrary{shapes}
\tikzset{
  dot hidden/.style={},
  line hidden/.style={},
  dot colour/.style={dot hidden/.append style={color=#1}},
  dot colour/.default=black,
  line colour/.style={line hidden/.append style={color=#1}},
  line colour/.default=black
}

\usepackage{xparse}
\NewDocumentCommand{\drawdie}{O{}m}{%
\begin{tikzpicture}[x=0.6em,y=0.6em,radius=0.13,#1]
  \draw[rounded corners=0.5,line hidden] (0,0) rectangle (1,1);
  \ifodd#2
    \fill[dot hidden] (0.5,0.5) circle;
  \fi
  \ifnum#2>1
    \fill[dot hidden] (0.2,0.2) circle;
    \fill[dot hidden] (0.8,0.8) circle;
   \ifnum#2>3
     \fill[dot hidden] (0.2,0.8) circle;
     \fill[dot hidden] (0.8,0.2) circle;
    \ifnum#2>5
      \fill[dot hidden] (0.8,0.5) circle;
      \fill[dot hidden] (0.2,0.5) circle;
     \ifnum#2>7
       \fill[dot hidden] (0.5,0.8) circle;
       \fill[dot hidden] (0.5,0.2) circle;
      \fi
    \fi
  \fi
\fi
\end{tikzpicture}%
}
